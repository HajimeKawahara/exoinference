\section{Quantities$\,^\ast$}
Symbols for solar systems are as follows. $\odot$: Sun, $J$: Jupiter, $\oplus$: Earth
$R_x$ indicates th radis of $x$, $M_x$ is its $x$.

\subsection*{Lengths and Angles}
The radii of the solar systems
\begin{itemize}
    \item $R_\odot = 7 \times 10^5$ km
    \item $R_J = R_\odot/10$
    \item $R_\oplus = R_J/10 = R_\odot/100$
\end{itemize}
and the Schwarzschild radius of the Sun is given by
\begin{itemize}
   \item $r_g = 2 G M_\odot/c^2$ = 3 km.
\end{itemize}
However, it is useful to remember the following relation,
\begin{itemize}
    \item $\displaystyle{\frac{r_g}{2 \, \mathrm{au}}} = 10^{-8}$.
\end{itemize}
The solar radius is 1/100 of 1 au,
\begin{itemize}
    \item $\displaystyle{\frac{2 R_\odot}{1 \, \mathrm{au}}} = 10^{-2}$.
\end{itemize}
The distance to the nearest star is approximately 1 pc. The facts for the angle of $1^{\prime \prime}$: 
\begin{itemize}
    \item $1^{\prime\prime} = 5 \times 10^{-6}$ radian
    \item $\displaystyle{\frac{1 \mathrm{au}}{1 \mathrm{pc}}} = 1^{\prime\prime}$ \\
    \item typical seeing $\sim 1^{\prime\prime}$
    \item the apparent diamter of the moon $\sim 30^{\prime}$
\end{itemize}

\subsection*{Weight}

\begin{itemize}
    \item $M_\odot = 2 \times 10^{33}$ g
    \item $M_J = M_\odot/1000$
    \item $M_\oplus = M_J/300$
\end{itemize}
