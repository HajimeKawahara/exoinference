\section{諸量$\,^\ast$}
天体の記号は$\odot$: 太陽、$J$: 木星、$\oplus$: 地球を表す。
$R_x$は天体$x$の半径、$M_x$は天体$x$の質量を表す。

\subsection*{長さと角度}
各天体の半径はだいたい
\begin{itemize}
    \item $R_\odot = 7 \times 10^5$ km
    \item $R_J = R_\odot/10$
    \item $R_\oplus = R_J/10 = R_\odot/100$
\end{itemize}
くらいである。太陽のシュワルツシルド半径
\begin{itemize}
   \item $r_g = 2 G M_\odot/c^2$ = 3 km
\end{itemize}
くらいだが、以下の便利な関係を使うとよい。
\begin{itemize}
    \item $\displaystyle{\frac{r_g}{2 \, \mathrm{au}}} = 10^{-8}$
\end{itemize}
太陽の直径は1 auの1/100である。
\begin{itemize}
    \item $\displaystyle{\frac{2 R_\odot}{1 \, \mathrm{au}}} = 10^{-2}$
\end{itemize}
一番近い星までの距離は 1 pcくらい. $1^{\prime \prime}$という角度について:
\begin{itemize}
    \item $1^{\prime\prime} = 5 \times 10^{-6}$ radian
    \item $\displaystyle{\frac{1 \mathrm{au}}{1 \mathrm{pc}}} = 1^{\prime\prime}$ \\
    \item 典型的なシーイング $\sim 1^{\prime\prime}$
    \item 月の視直径 $\sim 30^{\prime}$
\end{itemize}

\subsection*{重さ}

\begin{itemize}
    \item $M_\odot = 2 \times 10^{33}$ g
    \item $M_J = M_\odot/1000$
    \item $M_\oplus = M_J/300$
\end{itemize}
